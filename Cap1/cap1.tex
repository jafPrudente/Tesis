%------------------------------------------------------------
%	CAPITULO I
%------------------------------------------------------------

\chapter{Nombre del primer cap}
\label{capI}

\pagenumbering{arabic}

\begin{flushright}
	
	\small
	
	\textit{The best way to learn something} \\
	\textit{is to have to teach it.}
	
	\textsc{-John Archibald Wheeler}

    \vspace{10pt}
	
\end{flushright}


%------------------------------------------------------------
%	NOMBRE DE SECCIÓN
%------------------------------------------------------------

\section{Primer sección}
\label{sec:Cap1-1erseccion}

\index{Entrada importante}
\index{Concepto importante}
\index{Teorema! de la rata}
\index{Teorema! del perro}

\lipsum[1-5]

\begin{figure}[ht]
	\centering
	\includegraphics[width=1\textwidth]{\path Intro:functions.pdf}
	\caption{Este es el pie de figura UwUr.}
	\label{fig:Intro:functions}
\end{figure}

\lipsum[5-10]

Aquí hacemos algunas citas para probar formatos \cite{DiracElectrons}, \cite{Ramos-Sanchez}, \cite{Argelia-Tesis}, \cite{Degollado-Transparencias}.

\index{Teorema! de la serpiente}


%------------------------------------------------------------
%	NOMBRE DE SECCIÓN
%------------------------------------------------------------

\section{Segunda sección}
\label{sec:Cap1-2daseccion}

\lipsum[1-5]


%------------------------------------------------------------
%	NOMBRE DE SECCIÓN
%------------------------------------------------------------

\section{Tercer sección}
\label{sec:Cap1-3erseccion}

\lipsum[1-5]